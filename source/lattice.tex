\section{\label{sec:lattice}Lattice computation}

References: \cite{Hart2008,Hart2009,Cohen1993,CSE206A}

\subsection{Lattice}

\subsubsection{Choice of basis vectors of lattice}

The choice of basis vectors is not unique for a given lattice.
A lattice spanned by basis vectors $( \bm{a}_{1}, \dots, \bm{a}_{n} )$ and a lattice spanned by
\begin{align*}
  ( \bm{a}_{1}', \dots, \bm{a}_{n}' ) \coloneqq ( \bm{a}_{1}, \dots, \bm{a}_{n} ) \bm{P} \quad (\bm{P} \in \mathbb{R}^{n \times n})
\end{align*}
coincide if and only if $\bm{P} \in \mathrm{SL}(n, \mathbb{Z})$, where $ \mathrm{SL}(n, \mathbb{Z})$ is the set of $n \times n$ unimodular matrices.

\subsubsection{Delaunay reduction}

\subsection{Sublattice, Hermite normal form, and their applications}

\subsubsection{Sublattices and their equivalence}

A \term{sublattice} is a subset of lattice $L$ obtained by removing some lattice points from $L$\footnote{
  In other fields than mathematics and crystallography, a substructure of a crystal structure is called a ``sublattice'', and a superstructure of a crystal structure is called a ``superlattice''.
}.
A set of basis vectors of the sublattice is identified with the transformation matrix $\bm{M}$ such that the original set of basis vectors $\bm{A}$ is transformed into a new set of basis vectors $\bm{AM}$.
Therefore, the sublattice $L_{\bm{M}}$ is the set of lattice points expressed as
\begin{align}
  L_{\bm{M}} = \set{ \bm{AMn} }{ \bm{n} \in \mathbb{Z}^{3} }.
\end{align}
We refer to the absolute value of the determinant of $\bm{M}$, $\det \bm{M}$, as the index of the sublattice $L_{\bm{M}}$.
The index is identical to the number of lattice points in the sublattice $L_{\bm{M}}$.

\subsubsection{Hermite normal form}

Let $\bm{U}$ be a three-dimensional square unimodular matrix, where all elements are integers and $\det \, \bm{U} = \pm 1$.
Two matrices $\bm{M}$ and $\bm{MU}$ are equivalent in terms of lattice transformation.
This means that they derive the same sublattice expressed as
\begin{align}
    L_{\bm{M}} = L_{\bm{MU}}.
\end{align}

Their representative can be the canonical form called the \term{Hermite normal form} (HNF).
Any transformation matrix $\mathbf{M}$ can be converted to a unique form of the lower-triangular integer matrix, HNF, by multiplying the unimodular matrix $\mathbf{U'}$ from the right satisfying the relationship
\begin{equation}
    \label{eq:HNF}
    \mathbf{MU'} =
    \left(
    \begin{array}{ccc}
         a & 0 & 0 \\
         b & c & 0 \\
         d & e & f
    \end{array}
    \right),
\end{equation}
where $a > 0$, $0 \leq b < c$, $0 \leq d < f$, and $0 \leq e < f$.
% More mathematically it results from the fact $\mathbf{U}^{-1}$ is also a unimodular matrix.
% HNF
The requirement that diagonal elements $a$, $c$, and $f$ are all positive eliminates equivalent basis vectors obtained by inversion.
Also, the addition of a basis vector to another one or the subtraction of a basis vector from another one does not change the lattice itself.
Thus, we can choose remainders of $f$ as $d$ and $e$, and a remainder of $c$ as $b$.

Let us generalize HNFs to $m \times n$ integer matrix, $\bm{M}$.
It has a (column-style) Hermite normal form $\bm{H}$ if there exists a unimodular matrices $\bm{R} \in \mathbb{Z}^{n \times n}$ such that $\bm{H}=\bm{MR}$ satisfied the following conditions
\begin{enumerate}
    \item $H_{ij} \geq 0 \quad (1 \leq i \leq m, 1 \leq j \leq n)$
    \item $H_{ij} = 0 \quad (i < j , j > r)$
    \item $H_{ij} < H_{ii} \quad (i > j, 1 \leq i \leq r)$
    \item $r = \mathrm{rank} \bm{M}$
\end{enumerate}

If $\mathbf{M}$ is full rank, the Hermite normal form $\bm{H}$ is uniquely determined.

\subsubsection{Example to compute HNF}

\subsubsection{Union of lattices}

Just compute HNF!

\subsubsection{\label{sec:integer-linear-system}Integer linear system}

For given $\bm{A} \in \mathbb{Z}^{m \times n}$ and $\bm{b} \in \mathbb{Z}^{m}$, consider to solve integer linear system $\bm{Ax} = \bm{b}$ in $\bm{x} \in \mathbb{Z}^{n}$.
Let the Hermite normal form of $\bm{A}$ be $\bm{H} = \bm{AR}$, where $\bm{R}$ is unimodular and $\bm{H}$ is lower triangular.
The given linear system is
\begin{align}
  \begin{pmatrix}
    H_{11} &        & \bm{O} & \vdots     \\
    \vdots & \ddots &        & \bm{0} \\
    H_{r1} & \ldots & H_{rr} & \vdots     \\
    \ldots & \bm{0} & \ldots & \bm{O} \\
  \end{pmatrix}
  \bm{y} = \bm{b}
\end{align}
where $\mathbf{y} \coloneqq \bm{R}^{-1}\bm{x}$.
A special solution, $\bm{x}_{\mathrm{special}} = \bm{R}\bm{y}_{\mathrm{special}}$, is determined by Gaussian elimination if exists.
A general solution for $\bm{Hy}=\bm{0}$ is given by
\begin{align}
  \bm{y} = \begin{pmatrix} 0 \\ \vdots \\ 0 \\ n_{r+1} \\ \vdots \\ n_{m} \end{pmatrix}
        \quad (\forall n_{r+1},\dots, n_{m} \in \mathbb{Z}).
\end{align}

\subsection{Smith normal form and its applications}

\subsubsection{Distinct lattice points in sublattice}

When we consider the translational symmetry of a sublattice $L_{\bm{M}}$, two lattice points $\bm{m}, \bm{m}$ are equivalent if the distance between the two lattice points is a translation of $L_{\bm{M}}$, that is,
\begin{align}
  \label{eq:equiv-lattice-points}
  \bm{m} - \bm{m}' \in \bm{M}\mathbb{Z}^{3}.
\end{align}

The SNF of the transformation matrix $\mathbf{M}$ is useful to concretely write down Eq.~\eqref{eq:equiv-lattice-points}.
The SNF is one of the decompositions of an integer matrix $\mathbf{M}$ as
\begin{align}
  \label{eq:SNF}
  \bm{S} = \bm{PMQ},
\end{align}
where $\bm{P}$ and $\bm{Q}$ are unimodular matrices, and $\mathbf{S}$ is a diagonal integer matrix,
\begin{align}
  \bm{S} =
      \begin{pmatrix}
          S_{11} & 0 & 0 \\
          0 & S_{22} & 0 \\
          0 & 0 & S_{33}
      \end{pmatrix}.
\end{align}
Here $S_{11}$ is a divisor of $S_{22}$, and $S_{22}$ is a divisor of $S_{33}$.
We can rewrite Eq.~\eqref{eq:equiv-lattice-points} with Eq.~\eqref{eq:SNF} as
\begin{align}
  \label{eq:equiv-represents}
  [\bm{Pm}]_{\bm{S}} = [\bm{Pm}']_{\bm{S}},
\end{align}
where $[\cdot]_{\bm{S}}$ indicates to take modulus for the $i$th row by $S_{ii}$.
We mention that the range of $[\cdot]_{\mathbf{S}}$ is $\mathbb{Z}_{S_{11}} \times \mathbb{Z}_{S_{22}} \times \mathbb{Z}_{S_{33}}$ because a value of the $i$th row is a remainder by $S_{ii}$.

\subsubsection{Smith normal form}

Let $\bm{M}$ be $m \times n$ integer matrix.
There exist some unimodular matrices $\bm{L} \in \mathbb{Z}^{m \times m}$ and $\bm{R} \in \mathbb{Z}^{n \times n}$ such that

\begin{align}
    \bm{D}
    \coloneqq
    \bm{LMR}
    =
    \begin{pmatrix}
        d_{1}  &        & \bm{O} & \bm{0} \\
               & \ddots &        & \vdots \\
        \bm{O} &        & d_{r}  & \bm{0} \\
        \bm{0} & \cdots & \bm{0} & \bm{O}
    \end{pmatrix},
\end{align}
where $d_{i}$ is positive integer and $d_{i+1}$ devides $d_{i}$.
Then $\bm{D}$ is called Smith normal form.

\subsubsection{Extended Euclidean algorithm}

\url{https://twitter.com/tmaehara/status/1431205927353528321}

\subsubsection{Procedure to compute SNF}

\begin{align*}
  \begin{pmatrix}        2 & 4 & 4 \\        -6 & 6 & 12 \\        10 & -4 & -16 \\    \end{pmatrix}
  &\to    \begin{pmatrix}        2 & 0 & 0 \\        -6 & 18 & 24 \\        10 & -24 & -36 \\    \end{pmatrix}
  \to    \begin{pmatrix}        2 & 0 & 0 \\        0 & 18 & 24 \\        0 & -24 & -36 \\    \end{pmatrix}    \\
  &\to    \begin{pmatrix}        2 & 0 & 0 \\        0 & 18 & 6 \\        0 & -24 & -12 \\     \end{pmatrix}
  \to    \begin{pmatrix}        2 & 0 & 0 \\        0 & 18 & 6 \\        0 & 12 & 0 \\    \end{pmatrix}    \\
  &\to    \begin{pmatrix}        2 & 0 & 0 \\        0 & 6 & 18 \\        0 & 0 & 12 \\    \end{pmatrix}
  \to    \begin{pmatrix}        2 & 0 & 0 \\        0 & 6 & 0 \\        0 & 0 & 12 \\    \end{pmatrix}
\end{align*}

\subsubsection{Frobenius congruent}

For given $\bm{A} \in \mathbb{Z}^{m \times n}$ and $\bm{b} \in \mathbb{Z}^{m}$, consider to solve Frobenius congruent $\bm{Ax} \equiv \bm{b} \, (\mathrm{mod}\, \mathbb{R}/\mathbb{Z})$ for $\bm{x} \in \mathbb{R}^{n}$.
Let SNF of $\bm{A}$ be $\bm{D} = \bm{LAR}$, where $\bm{L}$ and $\bm{R}$ are unimodular matrices.
\begin{align}
  \bm{LAx} &= \bm{Lb} + \mathbb{Z}^{n} \\
  \bm{Dy}  &= \bm{v} + \mathbb{Z}^{n} \quad \mbox{where}\, \bm{y}:= \bm{R}^{-1} \bm{x},\, \bm{v}:= \bm{Lb} \\
  \bm{y}
    &= \begin{pmatrix} \frac{v_{1}}{D_{11}} \\ \vdots \\ \frac{v_{r}}{D_{rr}} \\ 0 \\ \vdots \\ 0 \end{pmatrix}
        + \begin{pmatrix} \frac{1}{D_{11}} n_{1} \\ \vdots \\ \frac{1}{D_{rr}} n_{r} \\ 0 \\ \vdots \\ 0 \end{pmatrix}
        + \begin{pmatrix} 0 \\ \vdots \\ 0 \\ a_{r+1} \\ \vdots \\ a_{m} \end{pmatrix}
        \quad (\forall n_{1}, \dots, n_{r} \in \mathbb{Z}, \forall a_{r+1}, \dots, a_{m} \in \mathbb{R})
\end{align}
