\section{\label{sec:normalizer}Normalizer}

The normalizer of a space group is a useful notion to consider the arbitrariness of coordinate transformations and devise classifications and algorithms for space groups.

\subsection{Euclidean normalizer and affine normalizer}

In general, a \term{normalizer} of a subgroup $H$ in a group $G$ is defined as
\begin{align}
  \mathcal{N}_{G}(H) \coloneqq \set{ g \in G }{ H^{g} = H }.
\end{align}
The subgroup $H$ is the normal subgroup of the normalizer, $H \trianglelefteq \mathcal{N}_{G}(H) \leq G$.

\begin{screen}
  \begin{defn}[Euclidean normalizer]
    The \term{Euclidean normalizer} of a space group $\mathcal{G}$ is a normalizer of $\mathcal{G}$ in $\mathcal{E}_{3}$,
    \begin{align}
      \mathcal{N}_{\mathcal{E}_{3}}(\mathcal{G}) \coloneqq \set{ (\bm{P}, \bm{p}) \in \mathcal{E}_{3} }{ (\bm{P}, \bm{p})^{-1} \mathcal{G} (\bm{P}, \bm{p}) = \mathcal{G} }.
    \end{align}
  \end{defn}
\end{screen}

\begin{screen}
  \begin{defn}[Affine normalizer]
    The \term{affine normalizer} of a space group $\mathcal{G}$ is a normalizer of $\mathcal{G}$ in $\mathcal{A}_{3}$,
    \begin{align}
      \mathcal{N}_{\mathcal{A}_{3}}(\mathcal{G}) \coloneqq \set{ (\bm{P}, \bm{p}) \in \mathcal{A}_{3} }{ (\bm{P}, \bm{p})^{-1} \mathcal{G} (\bm{P}, \bm{p}) = \mathcal{G} }.
    \end{align}
  \end{defn}
\end{screen}

The Euclidean and affine normalizers of a space group $\mathcal{G}$ may not be space groups: when the geometric crystal class of $\mathcal{G}$ is a pyroelectric point group, its normalizer has continuous translations along the polar direction.

The lists of Euclidean and affine normalizers of space groups are given in Sec.~3.5 of ITA \cite{ITA2016}.
When a translation lattice of $\mathcal{G}$ has a specialized metric, the Euclidean normalizer may have higher symmetry than other general cases.

\subsection{Wyckoff set and equivalent descriptions of crystal structure}

We define the Wyckoff position in Sec.~\ref{sec:conjugation_wyckoff} to classify points under the action of a space group.
However, the definition of the Wyckoff position depends on the coordinates system and two Wyckoff positions may be interchanged by a transformation.
For example, consider \ce{Ag3PO4} with $P\overline{4}3n$ (No. 218) \footnote{
  This example is adapted from Sec.~3.5.3.2 of ITA \cite{ITA2016}.
}.
Let $\mathcal{G}$ be a space-group representative of $P\overline{4}3n$.
The coset representatives of the Euclidean normalizer over $\mathcal{G}$ are
\begin{align*}
  \mathcal{N}_{\mathcal{E}_{3}}(\mathcal{G}) / \mathcal{G}:
    x,y,z;
    x + 1/2, y + 1/2, z + 1/2;
    \overline{x}, \overline{y}, \overline{z};
    \overline{x} + 1/2, \overline{y} + 1/2, \overline{z} + 1/2.
\end{align*}
Table~\ref{tab:equivalent-descriptions-example} shows four equivalent descriptions of the crystal structure obtained by the four coset representatives.
The Wyckoff positions $c$ and $d$ are interchanged by the inversion.

\begin{table}[htb]
  \begin{subtable}[h]{0.45\textwidth}
    \centering
    \begin{tabular}{ccc}
      \ce{P} & \ce{Ag} & \ce{O} \\
      \hline
      $2a$ & $6d$ & $8e$ \\
      \hline
      0 & 1/4 & $x=0.1486$ \\
      0 & 0   & $x$ \\
      0 & 1/2 & $x$ \\
    \end{tabular}
    \caption{$x, y, z$}
  \end{subtable}
  \hfill
  \begin{subtable}[h]{0.45\textwidth}
    \centering
    \begin{tabular}{ccc}
      \ce{P} & \ce{Ag} & \ce{O} \\
      \hline
      $2a$ & $6c$ & $8e$ \\
      \hline
      0 & 1/4 & $x=0.1486$ \\
      0 & 1/2 & $x$ \\
      0 & 0   & $x$ \\
    \end{tabular}
    \caption{$x + 1/2, y + 1/2, z + 1/2$}
  \end{subtable}
  \hfill
  \begin{subtable}[h]{0.45\textwidth}
    \centering
    \begin{tabular}{ccc}
      \ce{P} & \ce{Ag} & \ce{O} \\
      \hline
      $2a$ & $6d$ & $8e$ \\
      \hline
      0 & 1/4 & $x=-0.1486$ \\
      0 & 0   & $x$ \\
      0 & 1/2 & $x$ \\
    \end{tabular}
    \caption{$\overline{x}, \overline{y}, \overline{z}$}
  \end{subtable}
  \hfill
  \begin{subtable}[h]{0.45\textwidth}
    \centering
    \begin{tabular}{ccc}
      \ce{P} & \ce{Ag} & \ce{O} \\
      \hline
      $2a$ & $6c$ & $8e$ \\
      \hline
      0 & 1/4 & $x=-0.1486$ \\
      0 & 1/2 & $x$ \\
      0 & 0   & $x$ \\
    \end{tabular}
    \caption{$\overline{x} + 1/2, \overline{y} + 1/2, \overline{z} + 1/2$}
  \end{subtable}
  \caption{Equivalent descriptions of \ce{Ag3PO4} with $P\overline{4}3n$ resulting from its Euclidean normalizer.}
  \label{tab:equivalent-descriptions-example}
\end{table}

% Wyckoff sequence \cite{https://doi.org/10.1107/S0021889887086965}

The coarser classification of points that merges interchanged Wyckoff positions is called a \term{Wyckoff set}.
\begin{screen}
  \begin{defn}[Wyckoff set]
    Two points $\bm{x}$ and $\bm{y}$ belong to the same \term{Wyckoff set} for a space group $\mathcal{G}$ if their site-symmetry groups are conjugate subgroups of the affine normalizer $\mathcal{N}_{\mathcal{A}_{3}}(\mathcal{G})$.
  \end{defn}
\end{screen}
Here we define the equivalence under the affine normalizer not the Euclidean normalizer because the Euclidean normalizer depends on the translation lattice of $\mathcal{G}$ and make the classification complicated.
The list of Wyckoff sets is given in \href{https://www.cryst.ehu.es/html/cryst/wyckoffsets.html}{Bilbao Crystallographic Server}.

% \subsection{Normalizer action on vector system}
%
% Section 4.3 of Ref.~\cite{Souvignier08}

\subsection{Chirality}

\begin{screen}
  \begin{defn}
    An operation of the \term{first kind} is an operation with a proper linear part.
    An operation of the \term{second kind} is an operation with an improper linear part.
  \end{defn}
\end{screen}

\begin{screen}
  \begin{defn}[chiral]
    A crystal pattern is called \term{chiral} if it cannot be superposed on its inversion crystal pattern by an operation of the first kind.
    A space group of a chiral crystal pattern is called a \term{chiral space group}.
  \end{defn}
\end{screen}

Let $f$ be a crystal pattern and $\mathcal{G}$ be its space group.
When $f$ is not chiral, called achiral, there is an operation of the first kind, $(\bm{W}, \bm{w})$ with $\det \bm{W} > 0$ such that
\begin{align}
  \label{eq:chiral-crystal-patterns}
  (\bm{W}, \bm{w}) f = \overline{1} f.
\end{align}
Note that $(\bm{W}, \bm{w})$ does not necessarily belong to the space group of $f$ or $\overline{1}f$.

If $\mathcal{G}$ is a chiral space group, $\mathcal{G}$ should not have an operation of the second kind, $(\bm{W}', \bm{w}')$ with $\det \bm{W}' < 0$: otherwise, $\overline{1} \circ (\bm{W}', \bm{w}')$ satisfies Eq.~\eqref{eq:chiral-crystal-patterns}.
\begin{screen}
  \begin{defn}[Sohncke group]
    A space group is called a \term{Sohncke group} if it does not contain operations of the second kind.
  \end{defn}
\end{screen}
A chiral space group is a Sohncke group.
However, the converse does not hold.
In fact, if $\mathcal{N}_{\mathcal{E}_{3}}(\mathcal{G})$ contains an operation of the second kind, the space group is not chiral.
Thus, the chiral space group is a space group whose Euclidean normalizer contains only operations of the first kind, which is formed of 11 enantiomorphic pairs in three-dimensional space groups.



\subsection{Derivation of Euclidean normalizer}

We describe a method to derive the Euclidean normalizer in Ref.~\cite{Boisen:bx0358}.

\subsubsection{Basis vectors of Euclidean and affine normalizers}

If $( \bm{E}, \bm{p} ) \in \mathcal{N}_{\mathcal{E}_{n}}(\mathcal{G}) $, for all $(\bm{W}, \bm{w} ) \in \mathcal{G}$, there exists a symmetry operation $(\bm{W}', \bm{w}' ) \in \mathcal{G}$ such that
\begin{align}
  (\bm{E}, \bm{p}) (\bm{W}, \bm{w}) (\bm{E}, \bm{p})^{-1} = (\bm{W}', \bm{w}'),
\end{align}
where
\begin{align}
  \bm{W}' &= \bm{W} \\
  \bm{w}' &= \bm{p} + \bm{w} - \bm{Wp}.
\end{align}

When we take a primitive basis, the translation lattice is identified as $\mathbb{Z}^{n}$.
If $(\bm{W}, \bm{w} ), (\bm{W}, \bm{w}') \in \mathcal{G}$, we obtain
\begin{align}
  (\bm{W}, \bm{w}')^{-1} (\bm{W}, \bm{w}) = (\bm{E}, \bm{w} - \bm{w}' ) \in \mathcal{G}.
\end{align}
Thus, $\mathbf{w} - \mathbf{w}' \in \mathbb{Z}^{n}$.

In summary, the condition that $(\bm{E}, \bm{p} ) \in \mathcal{N}_{\mathcal{E}_{n}}(\mathcal{G})$ is equivalent to
\begin{align}
  \label{eq:normalizer-basis}
  (\bm{E} - \bm{W} ) \bm{p} \equiv \bm{0} ( \mathrm{mod} \, \mathbb{Z}^{n} )
\end{align}
for all $(\bm{W}, \bm{w}) \in \mathcal{G}$.
Conversely, a set of operations $(\bm{E}, \bm{p})$ satisfying Eq.~\eqref{eq:normalizer-basis} is a subgroup of $\mathcal{N}_{\mathcal{E}_{n}}(\mathcal{G})$.
The linear integer system of Eq.~\eqref{eq:normalizer-basis} can be solved by its Hermite or Smith normal form (see Sec.~\ref{sec:integer-linear-system}).
Because the above discussion is independent of whether an operation is an isometry or not, basis vectors of the Euclidean normalizer and affine normalizer can coincide.

% When the polar direction has no symmetry-equivalent directions, the normalizer contains continuous translations.
If the normalizer has a continuous translation along $\bm{v}$, $\bm{v}$ should be an eigenvector with eigenvalue 1 for all $\bm{W} \in \mathcal{P}(\mathcal{G})$.
The geometric crystal classes with such a polar direction are called \term{pyroelectric point groups}.
Then, Rotoinversions $\overline{3}$, $\overline{4}$, $\overline{6}$ cannot be contained in $\mathcal{P}(\mathcal{G})$ because they has no such an axis.
If $\mathcal{P}(\mathcal{G})$ has a mirror, its normal axis should be perpendicular to the primary rotation axis.
Thus, in three dimensions, the pyroelectric point groups are
\begin{itemize}
  \item $1$
  \item $2 (C_{2}), 3 (C_{3}), 4 (C_{4}), 6 (C_{6})$
  \item $m (C_{1v}), 2mm (C_{2v}), 3m (C_{3v}), 4mm (C_{4v}), 6mm (C_{6v})$.
\end{itemize}

\subsubsection{Linear part of Euclidean normalizer}

If $(\bm{P}, \bm{p} ) \in \mathcal{N}_{\mathcal{E}_{n}}(\mathcal{G})$, for all $(\bm{E}, \bm{w}) \in \mathcal{G}$, $(\bm{P}, \bm{p})$ satisfies the following condition at least
\begin{align}
  (\bm{P}, \bm{p}) (\bm{E}, \bm{w}) (\bm{P}, \bm{p})^{-1} = (\bm{E}, \bm{P}{w}) \in \mathcal{G}.
\end{align}
Therefore, $\bm{P} \in \mathrm{GL}_{n}(\mathbb{Z})$.
Similarly, considering $(\bm{P}, \bm{p} )^{-1} = (\bm{P}^{-1}, -\bm{P}^{-1}\bm{p} ) \in \mathcal{N}_{\mathcal{E}_{n}}(\mathcal{G})$, we obtain $\bm{P}^{-1} \in \mathrm{GL}_{n}(\mathbb{Z})$.
Thus, $\bm{P}$ is unimodular.
Moreover, $\bm{P}$ should belong to the Bravais group of the translation lattice, $\mathcal{B}(L)$, so that $(\bm{P}, \bm{p} )$ is an isometry.

\subsubsection{Translation part of Euclidean normalizer}

Let $\bm{P}$ be an element of Bravais group $\mathcal{B}(L)$.
If $(\bm{P}, \bm{p} ) \in \mathcal{N}_{\mathcal{E}_{n}}(\mathcal{G})$, the condition
\begin{align}
  (\bm{P}, \bm{p}) (\bm{W}, \bm{w}) (\bm{P}, \bm{p})^{-1} = (\bm{W}', \bm{w}') \in \mathcal{G}
\end{align}
gives
\begin{align}
    \label{eq:normalizer-translation-part}
    (\bm{E} - \bm{W}' ) \bm{p} \equiv \bm{w}' - \bm{Pw} \quad (\mathrm{mod}\, \mathbb{Z}^{n})
    \quad (\forall ( \bm{W}, \bm{w} ) \in \mathcal{G} ).
\end{align}
Conversely, a set of operations $(\bm{P}, \bm{p})$ satisfying Eq.~\eqref{eq:normalizer-translation-part} is a subgroup of $\mathcal{E}_{n}$.

\subsubsection{\label{sec:normalizer-example-p2mm}Example: normalizers of \texorpdfstring{$p2mm$}{p2mm}}

Let us compute the Euclidean normalizer of $p2mm$, $\mathcal{N}_{\mathcal{E}_{2}}(p2mm)$.
The generators of $p2mm$ are given in Eq.~\eqref{eq:p2mm}.

The linear integer system of Eq.~\eqref{eq:normalizer-basis} for $\bm{p} = \begin{pmatrix} p_{1} \\ p_{2} \end{pmatrix}$ is written as
\begin{align*}
  2 p_{1} &\equiv 0 \quad (\mathrm{mod} \, \mathbb{Z}) \\
  2 p_{2} &\equiv 0 \quad (\mathrm{mod} \, \mathbb{Z}).
\end{align*}
Thus, the basis vectors for the translation subgroup of $\mathcal{N}_{\mathcal{E}_{2}}(p2mm)$ is
$\left\{ \begin{pmatrix} \frac{1}{2} \\ 0 \end{pmatrix}, \begin{pmatrix} 0 \\ \frac{1}{2} \end{pmatrix} \right\}$.

The linear parts of $\mathcal{N}_{\mathcal{E}_{2}}(p2mm)$ is a subgroup of the Bravais group.
If the translations lattice does not have a specialized metric, the Bravais group is also $2mm$.
Then, there is no additional generator for the Euclidean normalizer.

% Specialized metric
If the translations lattice has a specialized metric, the Bravais group is $4mm$.
Then, the candidate for linear parts of additional generators is $\begin{pmatrix} 0 & -1 \\ 1 & 1 \\ \end{pmatrix}$, which is a fourfold rotation.
The integer linear system of Eq.~\eqref{eq:normalizer-translation-part} for a translation part $\bm{p} = \begin{pmatrix} p_{1} \\ p_{2} \end{pmatrix}$ gives
\begin{align*}
  \begin{pmatrix}
    2 & 0 \\
    0 & 2 \\
  \end{pmatrix}
  \begin{pmatrix} p_{1} \\ p_{2} \end{pmatrix}
  &\equiv \bm{0} \\
  \begin{pmatrix}
    2 & 0 \\
    0 & 0 \\
  \end{pmatrix}
  \begin{pmatrix} p_{1} \\ p_{2} \end{pmatrix}
  &\equiv \bm{0} \\
  \begin{pmatrix}
    0 & 0 \\
    0 & 2 \\
  \end{pmatrix}
  \begin{pmatrix} p_{1} \\ p_{2} \end{pmatrix}
  &\equiv \bm{0}.
\end{align*}
Thus, we can choose $\bm{p} = \bm{0}$ up to the translational symmetry of the Euclidean normalizer.
Then, $\overline{y},x$ is an additional generator of the Euclidean normalizer.
