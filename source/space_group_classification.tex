\section{\label{sec:classification}Classification of space groups}

\begin{tikzpicture}
  \tikzset{nodestyle/.style={draw}};

  \node[nodestyle] (family) {6 crystal families (Sec.~\ref{sec:crystal-family})};

  \node[nodestyle, below left=0.5cm of family] (lattice-system) {7 lattice systems (Sec.~\ref{sec:lattice-system})};
  \node[nodestyle, below=1cm of lattice-system] (bravais-class) {14 Bravais classes (Sec.~\ref{sec:bravais-class})};

  \node[nodestyle, below right=0.5cm of family] (crystal-system) {7 crystal systems (Sec.~\ref{sec:crystal-system})};
  \node[nodestyle, below=1cm of crystal-system] (geometric-class) {32 geometric crystal classes (Sec.~\ref{sec:geometric-class})};

  \node[nodestyle, below=3cm of family] (arithmetic-class) {73 arithmetic crystal classes (Sec.~\ref{sec:arithmetic-crystal-class})};
  \node[nodestyle, below=1cm of arithmetic-class] (affine) {219 affine space-group types (Sec.~\ref{sec:space-group-type})};
  \node[nodestyle, below=1cm of affine] (spacegroup) {230 crystallographic space-group types (Sec.~\ref{sec:space-group-type})};

  \draw (family) -- (lattice-system);
  \draw (family) -- (crystal-system);

  \draw (lattice-system) -- (bravais-class);
  \draw (crystal-system) -- (geometric-class);

  \draw (bravais-class) -- (arithmetic-class);
  \draw (geometric-class) -- (arithmetic-class);

  \draw (arithmetic-class) -- (affine);
  \draw (affine) -- (spacegroup);
\end{tikzpicture}

We classify space groups in three dimensions.

\subsection{\label{sec:space-group-type}Affine space-group type and space-group type}

It is natural to identify two space groups that are transformed into the another by changing coordinate systems.

\begin{screen}
  \begin{defn}[affine space-group type]
    Two space groups $\mathcal{G}, \mathcal{G}'$ belong to the same \term{affine space-group type} if they are conjugate by some affine mapping,
    \begin{align}
      \exists (\bm{P}, \bm{p}) \in \mathcal{A}_{3} \quad s.t. \quad (\bm{P}, \bm{p})^{-1} \mathcal{G} (\bm{P}, \bm{p}) = \mathcal{G}'.
    \end{align}
    These space groups are also called \term{affinely equivalent}.
  \end{defn}
\end{screen}

It is nontrivial fact that affine equivalence completely identifies the isomorphism of space groups.

\begin{screen}
  \begin{theorem}[Bieberbach]
    Two space groups are isomorphic if and only if they belong to the same affine type.
  \end{theorem}
\end{screen}

In crystallography, a tighter classification of space groups is often used.
That is, orientation-preserving affine mapping is only considered in transformations of coordinates systems\footnote{
  Some people indicate ``space groups'' as space-group types in their terminology.
  I think the clear distinction between space groups and space-group types makes our lives easier.
}.

\begin{screen}
  \begin{defn}[space-group type]
    Two space groups $\mathcal{G}, \mathcal{G}'$ belong to the same \term{(crystallographic) space-group type} if they are conjugate by some orientation-preserving affine mapping,
    \begin{align}
      \exists (\bm{P}, \bm{p}) \in \mathcal{A}^{+}_{3} \quad s.t. \quad (\bm{P}, \bm{p})^{-1} \mathcal{G} (\bm{P}, \bm{p}) = \mathcal{G}',
    \end{align}
    where orientation-preserving affine mapping is an affine mapping whose linear part is positive,
    \begin{align}
      \mathcal{A}^{+}_{n} \coloneqq \set{ \begin{pmatrix} \bm{W} & \bm{w} \\ \bm{0}^{\top} & 1 \end{pmatrix} }{ \bm{W} \in \mathrm{GL}(n, \mathbb{R}), \det \bm{W} > 0, \bm{w} \in \mathbb{R}^{n} }.
    \end{align}
  \end{defn}
\end{screen}

A pair of affine space-groups types that belong to the different crystallographic space-group types are called \term{enantiomorphic pair}.
For example, there are 11 enantiomorphic pairs in three dimensions as Table.~\ref{table:enantiomorphic-pairs-3d}.

\begin{table}[htb]
  \centering
  \caption{Enantiomorphic pairs in space groups}
  \label{table:enantiomorphic-pairs-3d}
  \begin{tabular}[h]{cc}
    $P 4_{1}$ (76)         & $P 4_{3}$ (78) \\
    $P 4_{1} 2 2$ (91)     & $P 4_{3} 2 2$ (95) \\
    $P 4_{1} 2_{1} 2$ (92) & $P 4_{3} 2_{1} 2$ (96) \\
    $P 3_{1}$ (144)        & $P 3_{2}$ (145) \\
    $P 3_{1} 1 2$ (151)    & $P 3{2} 1 2$ (153) \\
    $P 3_{1} 2 1$ (152)    & $P 3_{2} 2 1$ (154) \\
    $P 6_{1}$ (169)        & $P 6_{5}$ (173) \\
    $P 6_{2}$ (170)        & $P 6_{4}$ (172) \\
    $P 6_{1} 2 2$ (178)    & $P 6_{5} 2 2$ (179) \\
    $P 6_{2} 2 2$ (180)    & $P 6_{4} 2 2$ (181) \\
    $P 4_{3} 3 2$ (212)    & $P 4_{1} 3 2$ (213)
  \end{tabular}
\end{table}

\subsection{\label{sec:geometric-class}Geometric crystal class}

We consider classification based on a point group of a space group.

\begin{screen}
  \begin{defn}[geometric crystal class]
    Two subgroups of $\mathrm{GL}(3, \mathbb{Z})$, $\mathcal{P}$ and $\mathcal{P}'$, belong to the same \term{geometric crystal class} if they are conjugate by some invertible matrix,
    \begin{align}
      \exists \bm{P} \in \mathrm{GL}(3, \mathbb{R}) \quad s.t. \quad \bm{P}^{-1} \mathcal{P} \bm{P} = \mathcal{P}'.
    \end{align}
    Two space groups $\mathcal{G}$ and $\mathcal{G}'$ with point groups $\mathcal{P}$ and $\mathcal{P}'$, respectively, belong to the same \term{geometric crystal class} if $\mathcal{P}$ and $\mathcal{P}'$ belong to the same geometric crystal class.
  \end{defn}
\end{screen}

Two space groups belong to the same \term{Laue class} if point groups obtained by their point group with inversions belong to the same geometric crystal class (Table~\ref{tab:laue_class}).

\begin{table}[htb]
  \centering
  \caption{Laue classes in space groups}
  \label{tab:laue_class}
  \begin{tabular}[h]{l|l}
    \hline
    Laue class       & Geometric crystal class \\ \hline
    $\overline{1}$   & $1$, $\overline{1}$ \\
    $2/m$            & $2$, $m$, $2/m$ \\
    $mmm$            & $222$, $2mm$, $mmm$ \\
    $\overline{3}$   & $3$, $\overline{3}$ \\
    $\overline{3}m$  & $32$, $3m$, $\overline{3}m$ \\
    $4/m$            & $4$, $\overline{4}$, $4/m$ \\
    $4/mmm$          & $422$, $\overline{4}2m$, $4mm$, $4/mmm$ \\
    $6/m$            & $6$, $\overline{6}$, $6/m$ \\
    $6/mmm$          & $622$, $\overline{6}2m$, $6mm$, $6/mmm$ \\
    $m\overline{3}$  & $23$, $m\overline{3}$ \\
    $m\overline{3}m$ & $432$, $\overline{4}32$, $m\overline{3}m$ \\ \hline
  \end{tabular}
\end{table}

\todo{$3m1$ and $31m$}

\subsection{\label{sec:bravais-class}Bravais type of lattice and Bravais class}

We consider classification based on a translation lattice of a space group.

\begin{screen}
  \begin{defn}[translation lattice]
    A \term{translation lattice} $L$ of a space group $\mathcal{G}$ is a set of translation parts of translation subgroup of $\mathcal{G}$,
    \begin{align}
      L = \set{ \bm{t} }{ (\bm{I}_{3}, \bm{t}) \in \mathcal{G} }.
    \end{align}
  \end{defn}
\end{screen}

\begin{screen}
  \begin{defn}[Bravais group]
    A set of isometry mapping that pertains $L$ is called \term{Bravais group of $L$}.
    Letting $\bm{G}$ be a metric tensor of $L$, Bravais group of $L$ is
    \begin{align}
      \mathcal{B}(L) \coloneqq \set{ \bm{W} \in \mathrm{GL}(3, \mathbb{Z}) }{ \bm{W}^{\top}\bm{GW} = \bm{G} }.
    \end{align}
  \end{defn}
\end{screen}

We mention that a Bravais group may change as a set by transforming primitive basis.

\begin{screen}
  \begin{defn}[Bravais type of lattice]
    Two lattices $L$ and $L'$ belong to the same \term{Bravais type of lattice} if their Bravais groups are conjugate by some unimodular matrix,
    \begin{align}
      \exists \bm{P} \in \mathrm{SL}(3, \mathbb{Z}) \quad s.t. \quad \bm{P}^{-1} \mathcal{B}(L) \bm{P} = \mathcal{B}(L').
    \end{align}
  \end{defn}
\end{screen}
Remarkably, this definition of Bravais types of lattices is independent of choices for primitive basis.

\begin{screen}
  \begin{defn}[Bravais manifold]
    Let $K$ be a subgroup of $\mathrm{GL}(3, \mathbb{Z})$.
    A \term{space of metric tensors (Bravais manifold) of $K$} is the space of all metric tensors invariant with $K$,
    \begin{align}
      \bm{M}( K ) \coloneqq \set{ \bm{G} \in \mathbb{R}^{3 \times 3}_{\mathrm{sym}} }{ \bm{W}^{\top}\bm{G}\bm{W} = \bm{G} \quad (\forall \bm{W} \in K ) }
    \end{align}
  \end{defn}
\end{screen}

Let $L$ and $\mathcal{P}$ be a translation lattice and a point group of a space group $\mathcal{G}$, respectively.
When the dimension of $\bm{M}(\mathcal{B}(L))$ is smaller than that of $\bm{M}(\mathcal{P})$, the translation lattice $L$ is called to have \term{spaciealized metric}.

\todo{Add example of specialized metric and Euclidean normalizer}

\begin{screen}
  \begin{defn}[Bravais class]
    Let $L$ be a lattice with a metric tensor $\bm{G}$.
    A space group $\mathcal{G}$ with a point group $\mathcal{P}$ and a translation lattice $L$ belongs to a \term{Bravais class corresponding to the Bravais type of $L$} if $\bm{M}(\mathcal{P})$ and $\bm{M}(\mathcal{B}(L))$ are conjugate in $\mathrm{SL}(3, \mathbb{Z})$,
    \begin{align}
      \exists \bm{P} \in \mathrm{SL}(3, \mathbb{Z}) \,s.t.\, \bm{P}^{-1} \bm{M}(\mathcal{P}) \bm{P} = \bm{M}(\mathcal{B}(L)).
    \end{align}
  \end{defn}
\end{screen}

\todo{Note that the definition of Bravais classes is independent of whether a translation lattice is a specialized metric or not.}

\begin{screen}
  \begin{defn}[holohedry]
    A subgroup $\mathcal{P}$ of $\mathrm{GL}(3, \mathbb{Z})$ is called \term{holohedry} if there is a lattice $L$ whose Bravais group belongs to the same geometric crystal class of $\mathcal{P}$.
  \end{defn}
\end{screen}

\todo{limiting case from $mP$ to $oS$}

\subsection{\label{sec:arithmetic-crystal-class}Arithmetic crystal class}

\begin{screen}
  \begin{defn}[arithmetic crystal class]
    Two subgroups of $\mathrm{GL}(3, \mathbb{Z})$, $\mathcal{P}$ and $\mathcal{P}'$, belong to the same \term{arithmetic crystal class} if they are conjugate by some unimodular matrix,
    \begin{align}
      \exists \bm{P} \in \mathrm{SL}(3, \mathbb{R}) \quad s.t. \quad \bm{P}^{-1} \mathcal{P} \bm{P} = \mathcal{P}'.
    \end{align}
    Two space groups $\mathcal{G}$ and $\mathcal{G}'$ with point groups $\mathcal{P}$ and $\mathcal{P}'$, respectively, belong to the same \term{arithmetic crystal class} if $\mathcal{P}$ and $\mathcal{P}'$ belong to the same arithmetic crystal class.
  \end{defn}
\end{screen}

\begin{screen}
  \begin{defn}[Bravais arithmetic crystal class]
    The arithmetic crystal class of a space group $\mathcal{G}$ is called a \term{Bravais arithmetic crystal class} if the point group of $\mathcal{G}$ is the Bravais group of the translation lattice $L$ of $\mathcal{G}$,
    \begin{align}
      \mathcal{P} = \mathcal{B}(L).
    \end{align}
  \end{defn}
\end{screen}

Note that Bravais arithmetic crystal classes are classifications for arithmetic crystal classes of space groups, and Bravais classes are classifications for space groups.
The definition of Bravais arithmetic crystal classes is compatible with that of Bravais types of lattices (see Table~\ref{table:arithmetic-class-3d}).

\begin{table}[htb]
  \centering
  \caption{The correspondence of Bravais arithmetic crystal classes and Bravais types of lattices in three dimensions.}
  \label{table:arithmetic-class-3d}
  \begin{tabular}{l|l}
    \hline
    Bravais type of lattice & Bravais arithmetic crystal class \\ \hline
    $aP$                    & $\overline{1}P$ (2)              \\ \hline
    $mP$                    & $2/mP$ (10)                      \\
    $mC$                    & $2/mC$ (12)                      \\ \hline
    $oP$                    & $mmmP$ (47)                      \\
    $oS$                    & $mmmC$ (65)                      \\
    $oF$                    & $mmmF$ (69)                      \\
    $oI$                    & $mmmI$ (71)                   \\ \hline
    $tP$                    & $4/mmmP$ (123)                   \\
    $tI$                    & $4/mmmI$ (139)                   \\ \hline
    $hR$                    & $\overline{3}mR$ (166)           \\
    $hP$                    & $6/mmmP$ (191)                   \\ \hline
    $cP$                    & $m\overline{3}mP$ (221)          \\
    $cF$                    & $m\overline{3}mF$ (225)          \\
    $cI$                    & $m\overline{3}mI$ (229)          \\ \hline
  \end{tabular}
\end{table}

\todo{Example: $cm$ and $pm$}

\subsection{\label{sec:lattice-system}Lattice system}

\begin{screen}
  \begin{defn}[lattice system]
    Two lattices $L$ and $L'$ belong to the same \term{lattice system} if their Bravais groups belong to the same geometric crystal class,
    \begin{align}
      \exists \bm{P} \in \mathrm{SL}(3, \mathbb{Z}) \,s.t.\, \bm{P}^{-1} \mathcal{B}(L) \bm{P} = \mathcal{B}(L').
    \end{align}
  \end{defn}
\end{screen}

\todo{definition of lattice system on space groups}

Table~\ref{tab:lattice_system}

\begin{table}[htb]
  \centering
  \caption{Lattice systems in space groups}
  \label{tab:lattice_system}
  \begin{tabular}[h]{lll}
    \hline
    Lattice system & Holohedry & Bravias types of lattices \\ \hline
    Triclinic    & $\overline{1}$   & $aP$                   \\
    Monoclinic   & $2/m$            & $mP$, $mS$             \\
    Orthorhombic & $mmm$            & $oP$, $oS$, $oF$, $oI$ \\
    Tetragonal   & $4/mmm$          & $tP$, $tI$             \\
    Rhombohedral & $\overline{3}m$  & $hR$                   \\
    Hexagonal    & $6/mmm$          & $hP$                   \\
    Cubic        & $m\overline{3}m$ & $cP$, $cF$, $cI$       \\ \hline
  \end{tabular}
\end{table}

\subsection{\label{sec:crystal-system}Crystal system}

\todo{}

Two point groups belong to the same crystal system if and only if the sets of Bravais type of lattices on which these point groups act coincide.

Table~\ref{tab:crystal_system}

\begin{table}[htb]
  \centering
  \caption{Crystal systems in space groups}
  \label{tab:crystal_system}
  \begin{tabular}{ll}
    \hline
    Crystal system & Geometric crystal classes                                \\ \hline
    Triclinic      & $1, \overline{1}$                                        \\
    Monoclinic     & $2/m, m, 2$                                              \\
    Orthorhombic   & $mmm, mm2, 222$                                          \\
    Tetragonal     & $4/mmm, \overline{4}2m, 4mm, 422, 4/m, \overline{4}, 4$  \\
    Hexagonal      & $6/mmm, \overline{6}2m, 6mm, 622, 6/m, \overline{6}, 6$  \\
    Trigonal       & $\overline{3}m, 3m, 32, \overline{3}, 3$                 \\
    Cubic          & $m\overline{3}m, \overline{4}3m, 432, m\overline{3}, 23$ \\ \hline
  \end{tabular}
\end{table}

\subsection{\label{sec:crystal-family}Crystal family}
