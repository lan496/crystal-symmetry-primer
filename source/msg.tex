\section{\label{sec:msg}Magnetic space group}

\subsection{Definition}

% Magnetic space groups and magnetic symmetry operations
We consider a \term{time-reversal operation} $1'$ and call an index-two group generated from $1'$ as a \term{time-reversal group} $\{ 1, 1' \} \, (\cong \mathbb{Z}_{2})$, where $1$ represents an identity operation.
Let $\mathcal{M}$ be a subgroup of a direct product of $\mathrm{E}(3)$ and $\{ 1, 1' \}$.
An element $(\bm{W}, \bm{w})\theta$ of $\mathcal{M}$ is called a \term{magnetic symmetry operation}, where $\theta \in \{ 1, 1' \}$ is a \term{time-reversal part} of the magnetic symmetry operation.
A translation subgroup of $\mathcal{M}$ is defined similarly as
\begin{align}
  \mathcal{T}(\mathcal{M}) \coloneqq \set{ (\bm{E}, \bm{t}) }{ \exists \theta \in \{ 1, 1' \}, (\bm{E}, \bm{t})\theta \in \mathcal{M} }.
\end{align}
The subgroup $\mathcal{M}$ is called a \term{magnetic space group} (MSG) when its translation subgroup is generated from three independent translations.
We write a \term{magnetic point group} of $\mathcal{M}$ as
\begin{align}
    \mathcal{P}(\mathcal{M})
        &\coloneqq \set{ \bm{W}\theta }{ \exists \bm{w} \in \mathbb{R}^{3}, (\bm{W}, \bm{w})\theta \in \mathcal{M} }.
\end{align}

% Derived space groups
We consider two derived space groups from $\mathcal{M}$.
A \term{family space group} (FSG) of $\mathcal{M}$ is a space group obtained by ignoring time-reversal parts in magnetic symmetry operations:
\begin{align}
    \mathcal{F}(\mathcal{M}) \coloneqq \set{ (\bm{W}, \bm{w}) }{ \exists \theta \in \{ 1, 1' \}, (\bm{W}, \bm{w})\theta \in \mathcal{M} }.
\end{align}
A \term{maximal space subgroup} (XSG) of $\mathcal{M}$ is a space group obtained by removing antisymmetry operations:
\begin{align}
    \mathcal{D}(\mathcal{M}) \coloneqq \set{ (\bm{W}, \bm{w}) }{ (\bm{W}, \bm{w})1 \in \mathcal{M} }.
\end{align}

\subsection{Type of magnetic space group}

The MSGs are classified into the following four types:
\begin{itemize}
  \item (Type I)
    $\mathcal{M} = \mathcal{F}(\mathcal{M})1 = \mathcal{D}(\mathcal{M})1$:
    The MSG $\mathcal{M}$ does not have antisymmetry operations.
  \item (Type II)
    $\mathcal{M} = \mathcal{F}(\mathcal{M})1 \,\sqcup\, \mathcal{F}(\mathcal{M})1', \mathcal{F}(\mathcal{M}) = \mathcal{D}(\mathcal{M})$:
    The MSG $\mathcal{M}$ has antisymmetry operations and corresponding ordinary symmetry operations.
  \item (Type III)
    $\mathcal{M} = \mathcal{D}(\mathcal{M})1 \sqcup (\mathcal{F}(\mathcal{M}) \backslash \mathcal{D}(\mathcal{M})) 1'$ and $\mathcal{D}(\mathcal{M})$ is an index-two \term{translationengleiche} subgroup of $\mathcal{F}(\mathcal{M})$.
    Thus, translation subgroups of $\mathcal{F}(\mathcal{M})$ and $\mathcal{D}(\mathcal{M})$ are identical.
  \item (Type IV)
    $\mathcal{M} = \mathcal{D}(\mathcal{M})1 \sqcup (\mathcal{F}(\mathcal{M}) \backslash \mathcal{D}(\mathcal{M})) 1'$ and $\mathcal{D}(\mathcal{M})$ is an index-two \term{klessengleiche} subgroup of $\mathcal{F}(\mathcal{M})$.
    Thus, point groups of $\mathcal{F}(\mathcal{M})$ and $\mathcal{D}(\mathcal{M})$ are identical.
\end{itemize}

For a type-III MSG example, consider $\mathcal{M}_{\mathrm{rutile}} = \overline{P}4_{n}'2_{n}'$ (BNS number 136.498) in magnetic Hall symbols.
The FSG and XSG of $\mathcal{M}_{\mathrm{rutile}}$ are $\overline{P}4_{n}2_{n}$ (No. 136) and $\overline{P}22_{n}$ (No. 58) in Hall symbols, respectively.
% In magnetic Hall symbols, the original prime symbol ' in Hall symbols is replaced with the hat symbol $\hat{}$.

% n=abc, (1/2,1/2,1/2)
For a type-IV MSG example, consider $\mathcal{M}_{\mathrm{bcc}} = \overline{P}4231_{n}'$ (BNS number 221.97) in magnetic Hall symbols.
The FSG and XSG of $\mathcal{M}_{\mathrm{bcc}}$ are $\overline{I}423$ (No. 229) and $\overline{P}423$ (No. 221) in Hall symbols, respectively.

\subsection{BNS and OG symbols}

The BNS symbol represents each magnetic space-group type \cite{belov1957neronova}.
We refer to a setting of the BNS symbol as a BNS setting: For types-I, -II, and -III MSGs, it uses the same setting as the standard ITA setting of the FSG.
For type-IV MSG, it uses that of the XSG.

\todo{Add table to compare BNS and OG symbols}

\subsection{Magnetic Hall symbol}
