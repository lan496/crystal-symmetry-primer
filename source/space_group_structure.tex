\section{\label{sec:group_structure}Group structure of space groups}

\subsection{\label{sec:def_space_group}Definition of space group}

Let $f$ be a crystal pattern.
A space group of $f$ is defined as the stabilizer on $\mathcal{E}_{n}$.

\begin{screen}
  \begin{defn}[space group]
    The stabilizer of crystal pattern $f$ on $\mathcal{E}_{n}$
    \begin{align}
      \mathcal{G} \coloneqq \mathrm{Stab}_{\mathcal{E}_{n}} (f) = \set{ g \in \mathcal{E}_{n} }{ g \cdot f = f }
    \end{align}
    is called \term{space group} if $\mathcal{T}(\mathcal{G}) \coloneqq \mathcal{T}_{n} \cap \mathcal{G}$ is isomorphic to a $n$-dimensional lattice.
    For the space group $\mathcal{G}$, $\mathcal{T}(\mathcal{G})$ is called \term{translation subgroup} of $\mathcal{G}$.
  \end{defn}
\end{screen}

\subsection{Point group}

Let $\mathcal{G}$ be a space group.
The translation subgroup $\mathcal{T}(\mathcal{G})$ is normal subgroup of $\mathcal{G}$:
For any $ (\mathbf{W}, \mathbf{w}) \in \mathcal{G}$ and $(\mathbf{I}_{n}, \mathbf{t}) \in \mathcal{T}(\mathcal{G})$, one has
\begin{align}
  (\bm{W}, \bm{w}) (\bm{I}_{n}, \bm{t}) (\bm{W}, \bm{w})^{-1} = (\bm{I}_{n}, \bm{Wt}) \in \mathcal{T}(\mathcal{G}).
\end{align}

\begin{screen}
  \begin{defn}[point group]
    Let $\mathcal{G}$ be a space group.
    The set of the linear part of $\mathcal{G}$ is called \term{point group} of $\mathcal{G}$
    \begin{align}
      \mathcal{P} \coloneqq \set{ \mathbf{W} }{ (\bm{W}, \bm{w}) \in \mathcal{G} }.
    \end{align}
    The point group $\mathcal{P}$ is isomorphic to the factor group $\mathcal{G} / \mathcal{T}(\mathcal{G})$.
  \end{defn}
\end{screen}

We fix some basis vectors $(\bm{a}_{1}, \dots, \bm{a}_{n})$ of affine space, and act all mappings in $\mathcal{P}$ on the basis vectors.
Since mapping in the point group $\mathcal{P}$ is an isometry, it does not change the norm of vectors.
Therefore, there are only finitely many acted vectors because the maximum norm of acted vectors is bound by $\max_{i=1,\cdots,n} \mathbf{a}_{i}$.

\begin{screen}
  \begin{theorem}
    The point group $\mathcal{P}$ of a space group $\mathcal{G}$ is finite.
  \end{theorem}
\end{screen}

When we choose the lattice basis of $\mathcal{T}(\mathcal{G})$ as the basis vectors of the affine spaces, the translation subgroup $\mathcal{T}(\mathcal{G})$ is represented as $\set{ (\bm{I}_{n}, \bm{t}) }{ \bm{t} \in \mathbb{Z}^{n} }$.
And the action of point group $\mathcal{P}$ is described as integer matrices.

\subsection{Vector system}

In general, a space group is not specified only with a point group and translation subgroup.
We need to determine a translation part of each symmetry operation.

\begin{screen}
  \begin{defn}[vector system]
    Let $\mathcal{P}$ be a point group of a space group $\mathcal{G}$.
    The map $\bm{\tau}: \mathcal{P} \to \mathbb{R}^{n}$ is called a \term{vector system} if the map satisfies \term{cocycle condition}:
    \begin{align}
      \bm{\tau}(gh) = g \bm{\tau}(h) + \bm{\tau}(g) \, ( \mathrm{mod} \, \mathbb{Z}^{n}) \quad (\forall g, h \in \mathcal{P}).
    \end{align}
  \end{defn}
\end{screen}

We remark that we require the cocycle condition to have a vector system consistent with the product of two symmetry operations,
\begin{align*}
  \begin{pmatrix}
    g & \bm{\tau}(g) \\
    \bm{0}^{\top} & 1
  \end{pmatrix}
  \begin{pmatrix}
    h & \bm{\tau}(h) \\
    \bm{0}^{\top} & 1
  \end{pmatrix}
  =
  \begin{pmatrix}
    gh & g \bm{\tau}(h) + \bm{\tau}(g) \\
    \bm{0}^{\top} & 1
  \end{pmatrix}
\end{align*}

\todo{symmorphic and nonsymmorphic space groups}

\subsection{Working examples from plane groups}
